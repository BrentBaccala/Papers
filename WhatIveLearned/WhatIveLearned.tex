
\documentclass{article}

\title{What I've Learned about Quantum Mechanics}
\author{Brent Baccala}

\usepackage{amsmath}
\usepackage{amsfonts}

\usepackage{xcolor}
\usepackage{comment}
\usepackage{graphicx}

\usepackage[hidelinks]{hyperref}

\usepackage{tabularx}

\usepackage{longtable}

% For drawing ansatz diagrams

\usepackage{tikz}
\usetikzlibrary{calc}
\usetikzlibrary{positioning}
\usetikzlibrary{fit}
\usetikzlibrary{backgrounds}

\begin{document}
\parindent 0pt

\maketitle

\begin{abstract}
The author has spent twenty years trying to solve helium's ground state.
Here what I've learned.
\end{abstract}

%%\section*{The Algorithm}

\parskip 12pt

\subsection*{Introduction}

Nobody has ever found an exact solution to helium's ground state.
Nobody has ever proven that one does not exist.

What is a ``exact'' solution?  This could mean elementary functions,
liouvillian functions (elementary functions plus integrals), holonomic
functions (solutions of ODEs).  Holonomic functions are the most general,
and we have pretty good tools for testing if a holonomic function is
liovillian and/or elementary.

So, I regard holonomic functions as the holy grail; that's what I'm looking for
to ``solve'' helium's ground state.

\subsection*{The Risch Algorithm}

I first read about the Risch algorithm in the book ``A equals B''.

The Risch algorithm pretty much solves the problem of symbolic integration.
Either it finds an elementary solution to an integral or proves that none exists.
This isn't as useful as it might seem because there are plenty of non-elementary integrals,
but it at least answers a question we've had for three hundred years:
How do you ``do'' an integral?

The Risch algorithm excludes consideration of the absolute value and modulus functions
from consideration as elementary functions.  The modulus function fails to be analytic
anywhere and is the obvious complex extension of the absolute value function, which
fails to be differentiable at the origin.

Richardson's theorem tells us that if we introduce absolute value and modulus,
the integration problem rapidly becomes undecidable, so no extension of the
Risch algorithm is feasible.

A minor issue with the Risch algorithm is the decidability of the
constant subfield.  While we've proven that both $e$ and $\pi$ are transcendental,
it has never been proved that ${\mathcal Q}[e,\pi]$ has degree of transcendence two.
So, it's possible that some (large) polynomial in $e$ and $\pi$ could, in fact,
be zero.  The Risch algorithm requires us to test for constants being equal to zero.

Advanced calculus courses, covering multivariate calculus and complex analysis,
show us how to use those techniques to solve definite integrals.  We never
see them used to solve indefinite integrals, because there's no point.
If we could use multivariate or complex techniques to find an elementary
solution to an indefinite integral, then we could use the Risch algorithm
to obtain the same result.

\subsection*{Differential algebra}

Differential algebra has a reputation for being short on finding actual solutions.
The reason for this is that it only finds differential polynomial relationships,
not functions like sine and cosine, and furthermore that it only find such relationships
that hold for all solutions of a differential equation.

Further progress in differential algebra generally requires the introduction of
additional assumptions, especially for PDEs.  Consider
the Risch algorithm.  The additional assumption here is that the solution is elementary.

\subsection*{Liovuillian solutions}

$n^{\rm th}$-order linear ODEs have at most an $n$-dimensional solution space,
the determinant of the Wronkian matrix can be used to show that $n+1$ solutions
must be linearly dependent.  [Wikipedia on Wronkian, Singer and van der Put]
For a suitable class of solutions (complex analytic?), the solution space
will be exactly $n$ dimensional.

For second-order linear ODEs with rational (i.e, polynomial) coefficients,
Kovacic's algorithm gives us a constructive method to determine if
the solutions are Liovuillian.  Either all are or none are.
Extending Kovacic's algorithm to handle arbitrary rational coefficients
seems like low-hanging fruit; a nice Ph.D. project, I think.

For higher order (third order and higher) linear ODEs with rational coefficients, Singer and Bronstein
developed algorithms to find a Liovuillian subspace if it exists.
Factorization in the Weyl algebra is a useful technique here.

\subsection*{Partial Differential Equations}

The solutions to hydrogen's Schrodinger equation demonstrate that in general,
solution spaces to PDEs will not have a countable basis.

Consider the classical solutions to hydrogen, in the case where there
is no angular dependence.  Then all we have is a radical equation for $\Psi(r)$:

\[ r^2 \frac{d^2\Psi}{dr^2} + 2r \frac{d\Psi}{dr} + \left[2(E r^2+r)\right] \Psi = 0 \]

Basic ODE theory now tells us that for {\it any} value of $E$, this equation
will have a two-dimensional solution space, and we can express those solutions
using hypergeometric functions.  Only by enforcing the global square integrability
condition do we discard all but a countable infinite number of solutions, and
these are the classical solutions to hydrogen (the atomic shells).

Plus, the new pseudo-solution (a solution to the PDE that doesn't satisfy the global integrability condition)
that I discovered in January 2023 suggests that
there are many ODEs whose solutions will yield pseudo-solutions to the PDE.

This calls into
question the entire strategy of solving the differential equation first,
then applying the global integrability condition.  Without the global
condition, there are too many solutions to be easily categorized.

\end{document}
