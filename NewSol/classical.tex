
\documentclass{article}

\title{A New Solution of Hydrogen}
\author{Brent Baccala}

\usepackage{amsmath}

\usepackage{xcolor}
\usepackage{comment}
\usepackage{graphicx}

\usepackage[hidelinks]{hyperref}

\usepackage{tabularx}

\begin{document}
\parindent 0pt

Ideal \eqref{ideal:2} is also interesting.  Setting $d_1=1$ and $v_0=1$ (same logic as above), it simplies to:

\begin{equation}
\left(v_{3}, v_{2}, v_{1}, v_{0} - 1, n_{0} + 2, m_{1}, m_{0} + 2, d_{1} - 1, d_{0}, 2 E + n_{1}\right)
\end{equation}

\begin{equation}
\label{classical eq in ideal}
\begin{gathered}
v=r \\
v \Psi'' + 2 \Psi' + 2(1 + E v) \Psi = 0
\end{gathered}
\end{equation}

This is the classical radial equation obtained by seperation of variables\footnote{See
Pauling and Wilson or
\url{http://hyperphysics.phy-astr.gsu.edu/hbase/quantum/hydrad.html}}.  We use
spherical coordinates, set $\Psi = R(r)P(\theta)F(\psi)$, and obtain the above equation for $R(r)$,
though it is more commonly written in this form:

\begin{equation}
\frac{1}{R} \frac{d}{dr}\left[ r^2 \frac{dR}{dr}\right] + 2(Er^2 + r) = l(l+1)
\end{equation}

where $l$ is the orbital quantum number, which can be zero.  Set $l=0$, $v=r$ and $\Psi=R$ and
expand out the derivative to obtain \eqref{classical eq in ideal}.

%% A good question to ask is if any of these solutions are in $L^2$.  Remember that these are radial solutions,
%% so we expect to find our classical solutions in the bunch!

Mathematica finds solutions using hypergeometric series:

\includegraphics[page=1, clip, trim=1in 8.5in 1in 1in, width=\textwidth]{ideal 2.pdf}

{\tt Hypergeometric1F1} and {\tt HypergeometricU} are ${}_1F_1(a;b;z)$ and $U(a;b;z)$,
the confluent hypergeometric functions of the first and second kind,
respectively.  Both are solutions to Kummer's equation:

\begin{equation}
z\frac{d^2w}{dz^2} + (b-z)\frac{dw}{dz} - aw = 0
\end{equation}

Kummer's solution to this equation is ${}_1F_1(a;b;z)$:

\begin{equation}
{}_1F_1(a;b;z) = \sum_{n=0}^\infty \frac{a^{(n)}z^n}{b^{(n)}n!}
\end{equation}

(n) is the rising factorial.

Since Kummer's equation is a second-order linear ODE, we expect two linearly independent
solutions.  Tricomi's solution $U(a;b;z)$ not only solves Kummer's equation, but
for many values of $a$ and $b$, is linearly independent of Kummer's solution:

\begin{equation}
U(a;b;z) = \frac{\Gamma(1-b)}{\Gamma(a+1-b)} {}_1F_1(a;b;z) + \frac{\Gamma(b-1)}{\Gamma(a)} z^{1-b} {}_1F_1(a-1-b;2-b;z)
\end{equation}

Their presence here isn't too surprising.  A quick skim of Kunwar and van Hoeij,
{\it Second Order Differential Equations with Hypergeometric Solutions of Degree Three}
\url{https://www.math.fsu.edu/~hoeij/papers/issac13/2.pdf}
left me with the impression that many, but by no means all, second-order linear ODEs have
solutions with hypergeometric functions.

The major advantage possessed by all of these solutions, indeed the reason they are sought,
is that they are linear ODEs that solve a PDE.  In many ways, a hypergeometric series
is an even better form for the solution than a linear ODE.
We no longer need to use the second-order differential equation
form at all, as a series expansion is now available.  What are its convergence properties?
According to Mathematica public docs, it can be approximated to any precision.

I checked all energy levels from -10 to 10 in steps of 1/8.  For only these four did
Mathematica find any special simplification of the hypergeometric series:

\includegraphics[page=1, clip, trim=1in 5.75in 1in 3.25in, width=\textwidth]{ideal 2.pdf}

We see our classical solutions of $e^{-r}$ for energy $-\frac{1}{2}$ and $(2-r)e^{-r/2}$
for energy $-\frac{1}{8}$  How did we get $e^{-2r}$ for $-2$?

Why is $e^{-r}$ paired with some weird Ei integral?

The pattern: -1/2 maps to $e^{-r}$; -1/8 maps to $e^{-r/2}$; -1/18 maps to $e^{-r/3}$;
generalized as E maps to $e^{-\sqrt{2E}r}$.

The corresponding differential operator is $D+\sqrt{-2E}I$.  Squaring this we get $D^2 + 2\sqrt{-2E}D - 2E$.
This operator contains a $\sqrt{E}$ in the coefficient of the first-order term.  So we can't construct
a second-order differential operator that gives us $e^{-\sqrt{-2E}r}$ with multiplicity two unless we
allow $\sqrt{E}$ into our coefficients, which this ansatz does not.

We {\it can} however, pair $D+\sqrt{-2E}I$ with a different solution and obtain a suitable operator
that gives us the solution pair found by Mathematica.

Factoring in the Weyl algebra, we obtain:

\begin{equation}
vD^2 + 2D + (2+2Ev) I = (vD + (2-\sqrt{-2E}v)I) (D+\sqrt{-2E}I) + (2-2\sqrt{-2E})I
\end{equation}

so the factorization is only exact if E=-1/2.  In that case:

\begin{equation}
\label{Weyl algebra}
vD^2 + 2D + (2-v) I = (vD + (2-v)I) (D+I)
\end{equation}

Makes sense because only in the E=-1/2 case does a simple exponential solve the PDE.  In
higher energy cases we have to multiply by a Laguerre function like $(2-r)$.

Why isn't $(D+I)^2 = (D^2+2D+I)$ represented in our solution ideal?

Traditional methods\footnote{\url{https://dec41.user.srcf.net/h/IA_M/differential_equations/5_1}}
of solving second-order linear ODEs with constant coefficients tell us that degenerate
solutions like this yield both $e^{-r}$ and $re^{-r}$ as solutions.

So, we've introduced a new solution $y=re^{-r}$ that satisfies $Dy=e^{-r}-re^{-r}=(1-r)y$.

Is $D+(v-1)I$ a divisor of \eqref{Weyl algebra}?  How?

\end{document}
