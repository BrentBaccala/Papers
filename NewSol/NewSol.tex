
\documentclass{article}

\title{A New Solution of Hydrogen}
\author{Brent Baccala}

\usepackage{amsmath}

\usepackage{xcolor}
\usepackage{comment}
\usepackage{graphicx}

\usepackage[hidelinks]{hyperref}

\usepackage{tabularx}

\begin{document}
\parindent 0pt

\maketitle

\begin{abstract}
I show a previously unknown exact solution to the simplest time-independent Schr\"odinger equation for hydrogen.
The solution involves a Bessel function, is not separable, and is not in $L^2$.
\end{abstract}

\subsection*{Theorem}

Consider the following simple Sch\"odinger equation for the hydrogen atom:

\begin{equation}
\label{schrodinger}
-\frac{1}{2}\nabla^2 \Psi - \frac{1}{r}\Psi = E \Psi
\end{equation}

Let $J_0$ be the ordinary Bessel function $J_0$, and set

\begin{equation}
\label{solution}
\Psi = J_0(2\sqrt{x+r})
\end{equation}

where $x,y,z$ are Cartesian coordinates and $r=\sqrt{x^2+y^2+z^2}$.

\vskip 12pt

Then \eqref{solution} is an exact solution to \eqref{schrodinger}, with $E=0$.

\subsection*{Verification}

The result can be easily verified using Mathematica, as follows:

\includegraphics[page=1, clip, trim=1in 7in 1in 1in, width=\textwidth]{improved.pdf}

\vfill\eject
\subsection*{Proof}
The claim is that $\Psi = J_0(2\sqrt{x+r}) = (J_0 \circ 2\sqrt{v}) (x+r)$ satisfies:

\begin{equation}
\label{claim}
\left(\frac{\delta^2}{\delta^2 x} + \frac{\delta^2}{\delta^2 y} + \frac{\delta^2}{\delta^2 z}\right) \Psi + \frac{2}{r}\Psi = 0
\end{equation}

\vskip 12pt

Letting $v=x+r$, we compute the first partial derivatives of $\Psi$:

\begin{equation}
\begin{gathered}
\frac{\delta \Psi}{\delta x} = \frac{d v}{d x} \frac{d}{d v} \left(J_0 \circ 2\sqrt{v}\right) = \frac{d v}{d x} J_0'(2\sqrt{v}) v^{-1/2} \\
\frac{\delta \Psi}{\delta y} = \frac{d v}{d y} \frac{d}{d v} \left(J_0 \circ 2\sqrt{v}\right) = \frac{d v}{d y} J_0'(2\sqrt{v}) v^{-1/2} \\
\frac{\delta \Psi}{\delta z} = \frac{d v}{d z} \frac{d}{d v} \left(J_0 \circ 2\sqrt{v}\right) = \frac{d v}{d z} J_0'(2\sqrt{v}) v^{-1/2}
\end{gathered}
\end{equation}

\vskip 12pt

Next we compute the partial second derivatives of $\Psi$:

\begin{equation}
\label{second partials}
\begin{gathered}
\frac{\delta^2 \Psi}{\delta x^2} = \frac{d^2 v}{d x^2} J_0'(2\sqrt{v}) v^{-1/2} + \left(\frac{d v}{d x}\right)^2 J_0''(2\sqrt{v}) v^{-1} - \frac{1}{2} \left(\frac{d v}{d x}\right)^2 J_0'(2\sqrt{v}) v^{-3/2} \\
\frac{\delta^2 \Psi}{\delta y^2} = \frac{d^2 v}{d y^2} J_0'(2\sqrt{v}) v^{-1/2} + \left(\frac{d v}{d y}\right)^2 J_0''(2\sqrt{v}) v^{-1} - \frac{1}{2} \left(\frac{d v}{d y}\right)^2 J_0'(2\sqrt{v}) v^{-3/2} \\
\frac{\delta^2 \Psi}{\delta z^2} = \frac{d^2 v}{d z^2} J_0'(2\sqrt{v}) v^{-1/2} + \left(\frac{d v}{d z}\right)^2 J_0''(2\sqrt{v}) v^{-1} - \frac{1}{2} \left(\frac{d v}{d z}\right)^2 J_0'(2\sqrt{v}) v^{-3/2}
\end{gathered}
\end{equation}

\vskip 12pt

We need to know the derivatives of $v=r+x$ with respect to the coordinates:

\vskip 12pt

\begin{equation}
\label{first v}
\begin{gathered}
\frac{d v}{d x} = \frac{d}{d x} (x+r) = 1 + \frac{x}{r} \\
\frac{d v}{d y} = \frac{d}{d y} (x+r) = \frac{y}{r} \\
\frac{d v}{d z} = \frac{d}{d z} (x+r) = \frac{z}{r}
\end{gathered}
\end{equation}

\vskip 12pt

\begin{equation}
\label{second v}
\begin{gathered}
\frac{d^2 v}{d x^2} = \frac{d}{d x} \left(1 + \frac{x}{r}\right) = \frac{r - x(x/r)}{r^2} = \frac{r^2 - x^2}{r^3} \\
\frac{d^2 v}{d y^2} = \frac{r^2 - y^2}{r^3} \\
\frac{d^2 v}{d z^2} = \frac{r^2 - z^2}{r^3}
\end{gathered}
\end{equation}

\vskip 20pt

Substituting \eqref{first v} and \eqref{second v} into \eqref{second partials}, and \eqref{second partials}
into the LHS of \eqref{claim}, we obtain:

\begin{equation*}
\begin{aligned}
&\frac{d^2 v}{d x^2} J_0'(2\sqrt{v}) v^{-1/2} + \left(\frac{d v}{d x}\right)^2 J_0''(2\sqrt{v}) v^{-1} - \frac{1}{2} \left(\frac{d v}{d x}\right)^2 J_0'(2\sqrt{v}) v^{-3/2} \\
+& \frac{d^2 v}{d y^2} J_0'(2\sqrt{v}) v^{-1/2} + \left(\frac{d v}{d y}\right)^2 J_0''(2\sqrt{v}) v^{-1} - \frac{1}{2} \left(\frac{d v}{d y}\right)^2 J_0'(2\sqrt{v}) v^{-3/2} \\
+& \frac{d^2 v}{d z^2} J_0'(2\sqrt{v}) v^{-1/2} + \left(\frac{d v}{d z}\right)^2 J_0''(2\sqrt{v}) v^{-1} - \frac{1}{2} \left(\frac{d v}{d z}\right)^2 J_0'(2\sqrt{v}) v^{-3/2} \\
+& \frac{2}{r} J_0(2\sqrt{v})
\end{aligned}
\end{equation*}

\begin{equation*}
\begin{aligned}
=&\frac{r^2-x^2}{r^3} J_0'(2\sqrt{v}) v^{-1/2} + (1+\frac{x}{r})^2 J_0''(2\sqrt{v}) v^{-1} - \frac{1}{2} (1+\frac{x}{r})^2 J_0'(2\sqrt{v}) v^{-3/2} \\
&+ \frac{r^2-y^2}{r^3} J_0'(2\sqrt{v}) v^{-1/2} + \left(\frac{y}{r}\right)^2 J_0''(2\sqrt{v}) v^{-1} - \frac{1}{2} \left(\frac{y}{r}\right)^2 J_0'(2\sqrt{v}) v^{-3/2} \\
&+ \frac{r^2-z^2}{r^3} J_0'(2\sqrt{v}) v^{-1/2} + \left(\frac{z}{r}\right)^2 J_0''(2\sqrt{v}) v^{-1} - \frac{1}{2} \left(\frac{z}{r}\right)^2 J_0'(2\sqrt{v}) v^{-3/2} \\
&+ \frac{2}{r} J_0(2\sqrt{v})
\end{aligned}
\end{equation*}

\begin{equation*}
\begin{aligned}
=&\frac{3r^2-x^2-y^2-z^2}{r^3} J_0'(2\sqrt{v}) v^{-1/2} + (1+2\frac{x}{r} +\frac{x^2}{r^2} + \frac{y^2}{r^2} + \frac{z^2}{r^2}) J_0''(2\sqrt{v}) v^{-1} \\
&- \frac{1}{2} (1+2\frac{x}{r} +\frac{x^2}{r^2}+ \frac{y^2}{r^2} + \frac{z^2}{r^2}) J_0'(2\sqrt{v}) v^{-3/2}
+ \frac{2}{r} J_0(2\sqrt{v})
\end{aligned}
\end{equation*}

\begin{equation*}
=\frac{2}{r} J_0'(2\sqrt{v}) v^{-1/2} + (2+2\frac{x}{r}) J_0''(2\sqrt{v}) v^{-1} - \frac{1}{2} (2+2\frac{x}{r}) J_0'(2\sqrt{v}) v^{-3/2} + \frac{2}{r} J_0(2\sqrt{v})
\end{equation*}

\begin{equation*}
=\frac{2}{r} J_0'(2\sqrt{v}) v^{-1/2} + 2\frac{x+r}{r} J_0''(2\sqrt{v}) v^{-1} - \frac{x+r}{r} J_0'(2\sqrt{v}) v^{-3/2}
+ \frac{2}{r} J_0(2\sqrt{v})
\end{equation*}

Remembering that $v=x+r$,

%%\begin{equation*}
%%=\frac{2}{r} J_0'(2\sqrt{v}) v^{-1/2} + 2\frac{x+r}{r} J_0''(2\sqrt{v}) v^{-1} - \frac{1}{r} J_0'(2\sqrt{v}) v^{-1/2}
%%+ \frac{2}{r} J_0(2\sqrt{v})
%%\end{equation*}

\begin{equation*}
=\frac{2}{r} J_0'(2\sqrt{v}) v^{-1/2} + \frac{2}{r} J_0''(2\sqrt{v}) - \frac{1}{r} J_0'(2\sqrt{v}) v^{-1/2}
+ \frac{2}{r} J_0(2\sqrt{v})
\end{equation*}

\begin{equation*}
=\frac{2}{r} J_0''(2\sqrt{v}) + \frac{1}{r} J_0'(2\sqrt{v}) v^{-1/2} + \frac{2}{r} J_0(2\sqrt{v})
\end{equation*}

\begin{equation*}
=\frac{2}{r} J_0''(2\sqrt{v}) + \frac{2}{r\cdot2\sqrt{v}} J_0'(2\sqrt{v}) + \frac{2}{r} J_0(2\sqrt{v})
\end{equation*}

\begin{equation}
\label{last eq in derivation}
=\frac{2}{r} \left( J_0''(2\sqrt{v}) + \frac{1}{2\sqrt{v}} J_0'(2\sqrt{v}) + J_0(2\sqrt{v})\right)
\end{equation}

Now, the ordinary Bessel function $J_0(x)$ satisfies:

\begin{equation*}
x^2 J_0''(x) + xJ_0'(x) + x^2J_0(x) = 0
\end{equation*}

dividing through by $x^2$ and changing variables, we get:

\begin{equation*}
J_0''(2\sqrt{v}) + \frac{1}{2\sqrt{v}}J_0'(2\sqrt{v}) + J_0(2\sqrt{v}) = 0
\end{equation*}

which shows that \eqref{last eq in derivation} is zero, and establishes the proof of \eqref{claim}.

\subsection*{Generalization}
\parskip 12pt

The choice of $x$ is arbitrary, and any ordinary Bessel function can be used:

\begin{equation}
\label{generalized solution}
\Psi = F(2\sqrt{a_1 x+ a_2 y+ a_3 z+r})
\end{equation}

where

\begin{equation*}
a_1^2+a_2^2+a_3^2=1
\end{equation*}

and $F$ is any linear combination of the Bessel functions $J_0$ and $Y_0$.

\vskip 12pt

Any finite linear combination of functions of the form \eqref{generalized solution} also solves \eqref{schrodinger}.

\subsection*{Discovery}

% I used a Sage-based computer program with the following ansatz.

I found this solution roughly as follows.\footnote{
I discovered an alternate form of this solution using a somewhat more complex ansatz
on January 24, 2023.  By January 26, I had established the solution in its current form.
The original ansatz produced a rational function with
a 1254 term numerator and a 36 term denominator, that gave rise to a system of 224 equations.
}

Use Cartesian coordinates.  Let $v$ be a linear polynomial in the coordinates and the root $r=\sqrt{x^2+y^2+z^2}$,
with the following form (the $v_i$ are constants):

\begin{equation}
\label{v ansatz}
v = v_0 r + v_1 x + v_2 y + v_3 z
\end{equation}

Assume the solution to the input PDE \eqref{schrodinger}
is a linear second-order ODE w.r.t. $v$ with linear coefficeints,
with the following form:

\begin{equation}
\label{psi ansatz}
(d_0 + d_1 v) \frac{\delta^2\Psi}{\delta v^2} - (m_0 + m_1 v) \frac{\delta\Psi}{\delta v} - (n_0 + n_1 v) \Psi = 0
\end{equation}

or:

\begin{equation}
\label{psi ansatz sub}
(d_0 + d_1 v) \frac{\delta^2\Psi}{\delta v^2} = (m_0 + m_1 v) \frac{\delta\Psi}{\delta v} + (n_0 + n_1 v) \Psi
\end{equation}

Substituting \eqref{v ansatz} into \eqref{psi ansatz sub}, expanding derivatives in \eqref{schrodinger},
substituting the RHS of \eqref{psi ansatz sub} into \eqref{schrodinger} where the LHS of
\eqref{psi ansatz sub} appears,
replacing all instances of $r^2$ with $x^2+y^2+z^2$, and canceling GCDs, we obtain a rational function
with a 228 term numerator and an 18 term denominator.  We ignore the denominator.  The numerator begins:

\begin{equation}
%% -8r\Psi x^5 b_1 b_2^2 n_1 - r\Psi x^5 b_1 b_4^2 n_1 - r \Psi x^5 b_1 b_6^2 n_1 - 2r\Psi x^5 b_2 b_3 b_4 n_1 - \cdots
-2r\Psi x^3 E d_1 v_1 - 3r\Psi x^3 n_1 v_0^2 v_1 - r\Psi x^3 n_1 v_1^3 - r\Psi x^3 n_1 v_1 v_2^2 - r\Psi x^3 n_1 v_1 v_3^2 - \cdots
\end{equation}

We collect like terms in $x$, $y$, $z$, $r$, $\Psi$, and $\Psi'$, organizing the numerator like this:

\begin{equation}
%%\left(-8 b_1 b_2^2 n_1 - b_1 b_4^2 n_1 - b_1 b_6^2 n_1 - 2 b_2 b_3 b_4 n_1 - 2b_2 b_5 b_6 n_1 \right) r\Psi x^5 - \cdots
r\Psi x^3 \left(-2 E d_1 v_1 - 3 n_1 v_0^2 v_1 - n_1 v_1^3 - n_1 v_1 v_2^2 - n_1 v_1 v_3^2\right) - \cdots
\end{equation}

The expressions in parenthesis gives us a system of equations (only one is shown)
involving the $v_i$, $d_i$, $m_i$ and $n_i$ variables that, if satisfied,
will yield a solution to \eqref{schrodinger} in the form \eqref{v ansatz} and \eqref{psi ansatz}.  Once
duplicate equations are dropped, the system has 34 equations.

Several solution techniques are available to solve a system of polynomial equations; I used
a numerical approximation technique.  A laptop computer finds the following approximate solution
in less than three seconds:

\begin{equation}
\label{witness point}
\addtolength{\jot}{-3pt}
\begin{aligned}
&\verb!(E, 1.2793593235207163e-32)!\\
&\verb!(d0, 1.4231937528298923e-43)!\\
&\verb!(d1, 1.0)!\\
&\verb!(m0, -1.0)!\\
&\verb!(m1, 5.0839108285704363e-42)!\\
&\verb!(n0, -1.0)!\\
&\verb!(n1, -8.795561312674161e-33)!\\
&\verb!(v0, 1.0)!\\
&\verb!(v1, 0.47255672374941904)!\\
&\verb!(v2, 0.5379975369878418)!\\
&\verb!(v3, 0.6980320859632679)!
\end{aligned}
\end{equation}

This is a {\it witness point}, a term common in the literature, an approximate solution accurate enough
to recover an exact solution.

In this case, exactness recovery is simple and straightforward.  $E$, $d_0$, $m_1$ and $n_1$ are all quite small,
so we set them to zero, while $d_1$, $m_0$, $n_0$ and $v_0$ already have their exact values and:

\begin{equation*}
0.4725567237^2 + 0.5379975369^2 + 0.6980320859^2 \approx  1.0000000000
\end{equation*}

so we add $v_1^2 + v_2^2 + v_3^2 = 1$ to our solution and conclude that our witness point lies approximately
on the following algebraic variety:

\begin{equation}
\begin{gathered}
E = 0 \\
d_0 = 0 \qquad
d_1 = 1 \\
m_0 = -1 \qquad
m_1 = 0 \\
n_0 = -1 \qquad
n_1 = 0 \\
v_0 = 1 \qquad
v_1^2 + v_2^2 + v_3^2 = 1
\end{gathered}
\end{equation}

Substituting these values back into our ansatz, we conclude that $\Psi(v)$
is a solution of \eqref{schrodinger} under these conditions:

\begin{equation}
\label{related solution}
\begin{gathered}
v \frac{\delta^2\Psi}{\delta v^2} + \frac{\delta\Psi}{\delta v} + \Psi = 0 \\
v = v_1 x+ v_2 y+ v_3 z+r \\
v_1^2 + v_2^2 + v_3^2 = 1
\end{gathered}
\end{equation}

We now have to solve a second order ODE:

\begin{equation}
v \Psi''(v) + \Psi'(v) + \Psi(v) = 0
\end{equation}

Wolfram Mathematica\footnote{I originally used Wolfram Alpha} can now
analyze this equation and determine that it is equivalent
to the Bessel function \eqref{solution}.

\includegraphics[page=1, clip, trim=1in 9in 1in 1in, width=\textwidth]{find Bessel solution.pdf}

\subsection*{Software}

The program used to find the witness point is available here:

\centerline{\url{https://github.com/BrentBaccala/helium}}

It's a Sage script that works fine with Sage 9.0 on Ubuntu 20.

Use it to find the witness point \eqref{witness point} by
running Sage as follows:

\begin{verbatim}
load('helium.sage')     # loads the script
prep_hydrogen(5)        # select PDE:hydrogen and ansatz:5
multi_init()            # form the equation
multi_expand()          # expand out the numerator and collect like terms into a matrix
random_numerical()      # run numerical optimizer
\end{verbatim}

Here are some other convenient variables and functions in the script:

\begin{verbatim}
V, D, N, M              # trial forms of various polynomials
eq_a                    # PDE with ansatz substituted in and r^2 simplified
R                       # polynomial ring over integers
F                       # fraction field of R
F_eq_a                  # eq_a expanded out (in F)
F_eq_a_n                # expanded numerator (in R)
F_eq_a_d                # expanded denominator (in R)
eqns()                  # system of equations to solve
SciMin                  # solution from scipy.optimize.root
\end{verbatim}

\subsection*{Draft Status}

This paper is still a draft and is being updated regularly.

\subsection*{Contact}

The author maintains a discussion page for this result on his personal blog at:

\begin{center}
\small
\url{https://www.freesoft.org/blogs/soapbox/a-new-solution-of-hydrogen/}
\end{center}

\end{document}
