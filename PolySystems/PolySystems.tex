
\documentclass{article}

\title{Solving Systems of Polynomial Equations}
\author{Brent Baccala}

\usepackage{amsmath}
\usepackage{amsfonts}

\usepackage{xcolor}
\usepackage{comment}
\usepackage{graphicx}

\usepackage[hidelinks]{hyperref}

\usepackage{tabularx}

\usepackage{longtable}

% For drawing ansatz diagrams

\usepackage{tikz}
\usetikzlibrary{calc}
\usetikzlibrary{positioning}
\usetikzlibrary{fit}
\usetikzlibrary{backgrounds}

\begin{document}
\parindent 0pt

\maketitle

\begin{abstract}
The author has developed new software to solve systems of polynomial equations.
\end{abstract}

%%\section*{The Algorithm}

\parskip 12pt

%% \subsection*{Introduction}

Hans-Gert Gr\"abe --- On Factorized Gr\"obner Bases (1994)

Gr\"abe implemented his ideas in REDUCE package called CALI, which has since been ported
to Singular and is available as the Singular function {\tt facstd}.

Gr\"abe algorithm:

\begin{itemize}
\item try to factor each polynomial in a preprocessing step
\item during the Buchberger algorithm, try to factor each new reduced S-polynomial
\item when finished, apply tail reduction and check for factorization again
\end{itemize}

Singular algorithm:

\begin{itemize}
\item check for simple reductions in a preprocessing step
\item use the factoring Gr\"obner basis algorithm (with an option to disable the factorization) to complete the GTZ algorithm.
\end{itemize}

The ``simple reductions'' of the Singular algorithms are formed from polynomials in which
there is a variable that appears only once in the polynomial, and only as a single term with degree one.
Thus, $x$, $x-1$, and $x+y^2$ are all candidates (in $x$) for simple reduction, and
the reductions would be to substitute $x=0$, $x=1$, or $x=-y^2$ (respectively) into all
polynomials in the system.

These ``simple reductions'' are equivalent to reordering the variables
so that the most significant variable has a polynomial in
the system with a term consisting of that variable only,
then using this polynomial to apply a tail reduction step to the remaining polynomials.

The check for ``simple reductions'' is only done once, as a preprocessing step,
and not attempted again after the Gr\"obner basis calculation has begun.
Combined with the factoring Gr\"obner basis algorithm introduced by Gr\"abe,
the algorithm is seen to consist of attempting simple reductions, then
attempting factorization as the first step in the Gr\"obner basis algorithm.

I've found that the first factoring step often yields systems of equations in
which simple reductions can be found that were not present prior to
the factorization, so it seems profitable to check for further simple reductions.

My algorithm:

\begin{itemize}
\item check for simple reductions
\item try to factor each polynomial to split the system
\item repeat these two steps they yield no further reductions
\item use the Gr\"abe algorithm to apply the GTZ algorithm
\end{itemize}

Also, I do not split the system after each factorization, but rather factor
all of the polynomials in the system, and then perform a CNF-to-DNF conversion.

Prior implementations of the factorized Gr\"obner basis algorithm
store the polynomial systems in RAM.  Because I expect a large number
of systems to split off (on the order of millions), I store the
systems in a SQL database, using Python's pickling format.

The approach to the factorized Gr\"obner basis algorithm in \cite{Grabe06}
is described like this ($B$ is the system of polynomials to be solved):

\begin{quote}
During a preprocessing interreduce $B$ and try to factor each polynomial $f \in B$. If $f$ factors,
replace $B$ by a set of new problems, one for each factor of $f$. Update the side conditions
and apply the preprocessing recursively. This ends up with a list of interreduced problems
with non factoring base elements.
\end{quote}

This suggests that the factorization and splitting is performed independently on each
polynomial in the system.  If the system is large and many of the polynomials factor,
then this can lead to exponential growth in the number of systems.

This exponential growth in the number of systems can sometimes be mitigated
by first factoring all of the polynomials in the system, then performing
a logical transformation, as follows.

Consider a polynomial $f$ that factors as $f_1\cdot f_2\cdot f_3$.
The condition $f=0$ is then equivalent to the logical system ($f_1=0 \vee f_2=0 \vee f_3=0$).
If another polynomial $g$ factors as $g_1 \cdot g_2$, then the simultaneous
systems of equations $f=0; g=0$ is equivalent to this logical system:

\begin{equation}
(f_1=0 \vee f_2=0 \vee f_3=0) \wedge (g_1 = 0 \vee g_2 = 0)
\end{equation}

This logical system is in conjunctive normal form (CNF) and can be
converted to an equivalent system in disjunctive normal form (DNF), specifically:

\begin{equation}
\begin{split}
(f_1=0 \wedge g_1 = 0) \vee (f_2=0 \wedge g_1 = 0) \vee (f_3=0 \wedge g_1 = 0) \\
\vee (f_1=0 \wedge g_2 = 0) \vee (f_2=0 \wedge g_2 = 0) \vee (f_3=0 \wedge g_2 = 0)
\end{split}
\end{equation}

In this example, we have now obtained six simultaneous systems of polynomial equations,
each composed of irreducible polynomials.  A solution to any of them will solve the
original system.  The alegbraic variety defined by the original system is the union
of the six irreducible algebraic varieties defined by the factored systems.

The extended factorization step is to first factor all of the polynomials.  Then consider
each factor as a logical variable that is either true or false, depending on
whether the factor is or is not equal to zero.  Form a logical system in CNF
from the original polynomials and convert it to DNF to obtain systems of polynomials
formed from the factors.  Each system is further processed to analyze it.

For physically realistic problems, we hope that there is enough overlap between
the factorizations of the various polynomials to make this conversion desirable.

For example, one of the systems I'm concerned with (ansatz 16.6 for helium's ground state)
has 3486 polynomials.  2341 of them are irreducible; 969 of them factor into two irreducible factors, and 176 of them
factor into three irreducible factors.  Splitting on each of these factors independently,
the Gr\"abe algorithm would produce $2^{969} 3^{176} > 10^{375}$
distinct systems.  Yet there are only 2950 distinct factors, not $2341+2\cdot969+3\cdot176=4807$,
so there's a lot of overlap between the factors of the various polynomials.
Using CNF-to-DNF conversion, we obtain not $2^{969} 3^{176}$ systems, but only 48.

An obvious question is whether each CNF yields a unique DNF.  In general, the
answer is no, but the specific Boolean functions defined by polynomial systems are {\it monotone} Boolean functions,
because they do not involve the logical inverse of any logical variable, i.e,
they depend only on $f=0$ and not $f\ne 0$.  Monotone Boolean functions admit
a single unique DNF of unique prime implicants, due to a theorem stating that for monotone Boolean functions,
all prime implicants are essential.

Not only that, but a logical expression in DNF can be inverted to obtain a logical expression in CNF
for the inverse Boolean function.  Since the inverse of a monotone Boolean function is also
a monotone Boolean function, after a CNF-to-DNF conversion is performed on the inverse function,
the inverse function's DNF can be inverted to obtain the original CNF.  This procedure
is referred to in the literature as ``monotone dualization'', and is a very convenient
property for testing the correct operation of a software implementation.


\begin{thebibliography}{9}
\bibitem{Grabe06}
Gr\"abe (2006).  ``The Groebner Factorizer and Polynomial System Solving'',
Talk given at the Special Semester on Groebner Bases, Linz 2006.
\url{https://www.ricam.oeaw.ac.at/specsem/srs/groeb/download/06_02_Solver.pdf}
\end{thebibliography}

\end{document}
