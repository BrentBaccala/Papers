
\documentclass{article}

\title{Solving Systems of Polynomial Equations}
\author{Brent Baccala}

\usepackage{amsmath}
\usepackage{amsfonts}

\usepackage{xcolor}
\usepackage{comment}
\usepackage{graphicx}

\usepackage[hidelinks]{hyperref}

\usepackage{tabularx}

\usepackage{longtable}

% For drawing ansatz diagrams

\usepackage{tikz}
\usetikzlibrary{calc}
\usetikzlibrary{positioning}
\usetikzlibrary{fit}
\usetikzlibrary{backgrounds}

\begin{document}
\parindent 0pt

\maketitle

\begin{abstract}
The author has developed new software to solve systems of polynomial equations.
\end{abstract}

%%\section*{The Algorithm}

\parskip 12pt

%% \subsection*{Introduction}

Hans-Gert Gr\"abe --- On Factorized Gr\"obner Bases (1994)

Gr\"abe implemented his ideas in REDUCE package called CALI.

Gr\"abe algorithm:

\begin{itemize}
\item try to factor each polynomial in a preprocessing step
\item try to factor each new reduced S-polynomial
\item when finished, apply tail reduction and check for factorization again
\end{itemize}

Singular algorithm:

\begin{itemize}
\item check for simple reductions in a preprocessing step
\item use the factoring Gr\"obner basis algorithm (with an option to disable the factorization)
\end{itemize}

The ``simple reductions'' of the Singular algorithms are equivalent to reordering the variables,
then applying a tail reduction step.

\end{document}
